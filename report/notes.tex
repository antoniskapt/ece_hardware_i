\section{Παρατηρήσεις}

\subsection{Αρχείο καταχωρητών}
Σχετικά με το αρχείο καταχωρητών, ο πρώτος καταχωρητής, $\text{x}0$, στην αρχιτεκτονική RISC είναι βραχυκυκλωμένος στη γείωση. Κατα αυτόν τον τρόπο διατηρεί πάντα σταθερή τιμή μηδέν. Εφόσον κάτι τέτοιο δεν αναφέρεται στις προδιαγραφές της εκφώνησης δεν υλοποιείται. Στην περίπτωση που θα απαιτούνταν η υλοποίηση της προδιαγραφής αυτής θα μπορούσε να γίνει με δύο τρόπους. Είτε με μία διαρκή ανάθεση της τιμής μηδέν στον καταχωρητή $\text{x}0$, είτε με έναν έλεγχο της διεύθυνσης εγγραφής. Εάν η διεύθυνση εγγραφής είναι ο καταχωρητής $\text{x}0$, τότε η εγγραφή αγνοείται.

\begin{center}
\begin{lstlisting}[language=Verilog,tabsize=2,caption={Τροποποίηση του αρχείου καταχωρητών ώστε ο $\text{x}0$ να είναι βραχυκυκλωμένος στη γείωση.},numbers=left,firstnumber=66,basicstyle=\ttfamily,xleftmargin=2em]
always @(posedge clk) begin
	if(write) begin
		if(writeReg!=0)
			x_reg[writeReg]<=writeData;
	end
end
\end{lstlisting}
\end{center}

\subsection{Compilation}
Για τη μεταγλώττιση του source code και την παραγωγή των dump files με τις τιμές για την προβολή των χρονοδιαγραμμάτων έχει προστεθεί το αρχείο \texttt{src/makefile}. Το target \texttt{all} κάνει compile το testbench της αριθμομηχανής και το testbench του επεξεργστή. Το target \texttt{calc} μεταγλωττίζει μόνο το testbench της αριθμομηχανής και το target \texttt{top} μεταγλωττίζει μόνο το testbench του επεξεργαστή. Υπάρχουν δύο επιπλέον targets για τη διαγραφή των αναπαράξιμων αρχείων.\par